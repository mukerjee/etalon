% TEMPLATE for Usenix papers, specifically to meet requirements of
%  USENIX '05
% originally a template for producing IEEE-format articles using LaTeX.
%   written by Matthew Ward, CS Department, Worcester Polytechnic Institute.
% adapted by David Beazley for his excellent SWIG paper in Proceedings,
%   Tcl 96
% turned into a smartass generic template by De Clarke, with thanks to
%   both the above pioneers
% use at your own risk.  Complaints to /dev/null.
% make it two column with no page numbering, default is 10 point

% Munged by Fred Douglis <douglis@research.att.com> 10/97 to separate
% the .sty file from the LaTeX source template, so that people can
% more easily include the .sty file into an existing document.  Also
% changed to more closely follow the style guidelines as represented
% by the Word sample file. 

% Note that since 2010, USENIX does not require endnotes. If you want
% foot of page notes, don't include the endnotes package in the 
% usepackage command, below.

% This version uses the latex2e styles, not the very ancient 2.09 stuff.
% \documentclass[letterpaper,twocolumn,10pt]{article}
\pdfminorversion=4

\documentclass{hotnets15}

\usepackage{epsfig,endnotes}
\usepackage{color}
\usepackage{colortbl}
\usepackage{verbatim}
\usepackage[font=bf]{caption}
\usepackage{xspace}
\usepackage{mdframed}
\usepackage{float}
\usepackage{microtype}
\usepackage{newtxmath}

\usepackage{tikz}
\usepackage{enumitem}
\usepackage{booktabs}  % for toprule, bottomrule in tables
\usepackage[sort,space]{cite}
\usepackage{flushend}  % for balanced final columns


\usepackage{graphicx}
\usepackage{caption}
\usepackage{subfigure}

\newcommand{\subparagraph}{}
\usepackage[compact]{titlesec}

\usepackage{url}


\newcommand{\tightcaption}[1]{\vspace{-0.3cm}\caption{#1}\vspace{-0.3cm}}
\newcommand{\tightsection}[1]{\section{#1}}
\newcommand{\tightsubsection}[1]{\subsection{#1}}
\newcommand{\tightsubsubsection}[1]{\subsubsection{#1}}

\newcommand{\eg}{{\it e.g.,}\xspace}
\newcommand{\ie}{{\it i.e.,}\xspace}

\newcounter{note}[section]
\renewcommand{\thenote}{\thesection.\arabic{note}}

\newcommand{\Section}{\S}

\usepackage{pifont}
\newcommand{\cmark}{\ding{51}}%
\newcommand{\xmark}{\ding{55}}%

\newcommand{\fillme}{{\bf XXX}~}

\definecolor{purple}{RGB}{175,0,255}


\newcommand{\mypara}[1]{\medskip\noindent{\bf {#1}:}~}
\newcommand{\myparatight}[1]{\smallskip\noindent{\bf {#1}:}~}
\newcommand{\myparaq}[1]{\smallskip\noindent{\bf {#1}?}~}
\newcommand{\mysubparatight}[1]{\noindent {\it {#1}:}~}

%-- place any standard commands/environments here to get included in
%-- documents.  When you include this file, you should do it before
%-- the \begin{document} tag.

%%%%%%%%%%%%%%%%%%%%%%%%%%%%%%%%%%%%%%%%%%%%%%%%%%%%%%%%%%%%%%%%%%%%%%
%-- CHANGES:
%-- 07/31/01 -jstrunk- Added command to set the paper margins.

%-- Provides fixed width font for commands and code snips.
\newcommand{\code}[1]{\texttt{\textbf{#1}}}

%-- Terms...  Use this to introduce a term in the paper.
\newcommand{\term}[1]{\emph{#1}}

%-- Provides stylization for e-mail addresses
%\newcommand{\email}[1]{\emph{(#1)}}

%-- Starts a minor section (puts the title inline w/ the text.
\newcommand{\minorsection}[1]{\textbf{#1}:}

%-- Jiri caption
\newcommand{\minicaption}[2]{\caption[#1]{\textbf{#1.} #2}}

%-- Units on numbers: 4KB -> \units{4}{KB}
\newcommand{\units}[2]{#1~#2}

%-- Commands...  i.e. WRITE commands.
\newcommand{\command}[1]{{\sc \MakeLowercase{#1}}}

%-- For notes about things that need to be fixed.
\newcommand{\fix}[1]{\marginpar{\LARGE\ensuremath{\bullet}}
    \MakeUppercase{\textbf{[#1]}}}
%-- For adding inline notes to a draft preceded by your initials
%-- E.g., \fixnote{JJW}{What the heck is a foobar?}
\newcommand{\fixnote}[2]{\marginpar{\LARGE\ensuremath{\bullet}}
    {\textbf{[#1:} \textit{#2\,}\textbf{]}}}

%-- Setting margins: \setmargins{left}{right}{top}{bottom}
\newcommand{\setmargins}[4]{
    % Calculations of top & bottom margins
    \setlength\topmargin{#3}
    \addtolength\topmargin{-.5in}  %-- seems like this should be 1, but .5
                                   %-- balances the text top to bottom
    \addtolength\topmargin{-\headheight}
    \addtolength\topmargin{-\headsep}
    \setlength\textheight{\paperheight}
    \addtolength\textheight{-#3}
    \addtolength\textheight{-#4}

    % Calculations of left & right margins
    \setlength\oddsidemargin{#1}
    \addtolength\oddsidemargin{-1in}
    \setlength\evensidemargin{\oddsidemargin}
    \setlength\textwidth{\paperwidth}
    \addtolength\textwidth{-#1}
    \addtolength\textwidth{-#2}
}

%-- For the tabularx environment... Using L, C, R as the column type
%-- will left, center, or right justify the text.
\newcolumntype{L}{X}
\newcolumntype{C}{>{\centering\arraybackslash}X}
\newcolumntype{R}{>{\raggedleft\arraybackslash}X}

%-- To comment out a swatch of text, use \omitit{blah blah blah}
\long\def\omitit#1{}

%-- Inline title; useful for sub-sub-sections in which you don't want a separate
%-- line for the title.
\newcommand{\inlinesection}[1]{\smallskip\noindent{\textbf{#1.}}}


%%%
%%% COMMENTS / TODOS
%%%
\usepackage{ifthen}
\usepackage{xcolor}
\newcommand{\exclude}[1]{}
\newcommand{\showComments}{yes}
\newcommand{\note}[2]{
    \ifthenelse{\equal{\showComments}{yes}}{\textcolor{#1}{#2}}{}
}
\newcommand{\TODO}[1]{%
	\addcontentsline{tdo}{todo}{\protect{#1}}%
	\note{red}{TODO: #1}
}

\makeatletter \newcommand{\listoftodos}
{\section*{Todo List} \@starttoc{tdo}}
\newcommand{\l@todo}
{\@dottedtocline{1}{0em}{2.3em}} \makeatother


\newcommand{\myparaittight}[1]{\smallskip\noindent{\emph {#1}:}~}
\newcommand{\question}[1]{\smallskip\noindent{\emph{Q:~#1}}\smallskip}
\newcommand{\myparaqtight}[1]{\smallskip\noindent{\bf {#1}}~}
\newcommand{\jc}[1]{\note{green}{[HZ: #1]}}
\newcommand{\matt}[1]{\note{blue}{[MKM: #1]}}
\newcommand{\srini}[1]{\note{magenta}{[SS: #1]}}
\newcommand{\bruce}[1]{\note{red}{[BMM: #1]}}
\newcommand{\nadi}[1]{\note{purple}{[INB: #1]}}
\newcommand{\mycomment}[1]{}

\newcounter{packednmbr}

\newenvironment{packedenumerate}{\begin{list}{\thepackednmbr.}{\usecounter{packednmbr}\setlength{\itemsep}{0.5pt}\addtolength{\labelwidth}{-4pt}\setlength{\leftmargin}{\labelwidth}\setlength{\listparindent}{\parindent}\setlength{\parsep}{1pt}\setlength{\topsep}{0pt}}}{\end{list}}

\newenvironment{packeditemize}{\begin{list}{$\bullet$}{\setlength{\itemsep}{0.5pt}\addtolength{\labelwidth}{-4pt}\setlength{\leftmargin}{\labelwidth}\setlength{\listparindent}{\parindent}\setlength{\parsep}{1pt}\setlength{\topsep}{0pt}}}{\end{list}}


%
\def\sharedaffiliation{%
\end{tabular}
\begin{tabular}{c}}
%

\begin{document}

\title{Characterizing Reconfigurable Topologies in Datacenter Fabrics}

\numberofauthors{1}
\author{\alignauthor Paper \#16, 7 pages}


\maketitle



\subsection*{Abstract}
Increasing pressure for higher throughput, better response times, and lower cost
in datacenters have pushed researches to explore alternative technologies for
network fabrics (e.g., 60GHz wireless, free-space optics, optical circuit
switching). While these alternatives can theoretically meet the demand, they all
share a common limitation: which nodes can communicate at a given time are
limited due to long switching times. These constraints ``cut'' links for brief
periods, reconfiguring the topology. 

Applications in datacenters continuously push the boundaries of what the network
can provide



Recently there have been many exciting technologies proposed for use as
datacenter fabrics (e.g., 60GHz wireless, free-space optics, optical circuit
switches), each with their own unique technical constraints. These technologies
all share a common limitation: which nodes can communicate at a given time is
constrained; changing which nodes can communicate requires (automated) topology
reconfiguration. Proper reconfiguration of the topology is key to provide
reasonable performance. Prior work treat each of these new network technologies
as point solutions, with little focus on how each relate to each other in the
broader datacenter context. The goal of this line of work is to provide an
generalized framework that each of these new technologies naturally fits into
that can optimize network usage based on the technology's specific needs. This
paper takes the first steps towards this goal by characterizing the space in
which these technologies reside, as well as looking at how network-level
primitives (e.g., multicast, anycast) in addition to cross-layer optimization
can greatly improve application completion time over these technologies. Our
initial simulation results show that for certain workloads we can achieve an
xx\% decrease in application completion time over a naive baseline without
modifying applications and a xxx\% decrease in completion time with application
modification.


%% A range of new datacenter switch designs combine wireless or optical circuit
%% technologies with electrical packet switching to deliver higher performance at
%% lower cost than traditional packet-switched networks.  These ``hybrid'' networks
%% schedule large traffic demands via a high-rate circuits and remaining traffic
%% with a lower-rate, traditional packet-switches. Achieving high utilization
%% requires an efficient scheduling algorithm that can compute proper circuit
%% configurations and balance traffic across the switches. Recent proposals,
%% however, provide no such algorithm and rely on an omniscient oracle to compute
%% optimal switch configurations.

%% Finding the right balance of circuit and packet switch use is difficult:
%% circuits must be reconfigured to serve different demands, incurring non-trivial
%% switching delay, while the packet switch is bandwidth constrained. Adapting
%% existing crossbar scheduling algorithms proves challenging with these
%% constraints. In this paper, we formalize the hybrid switching problem, explore
%% the design space of scheduling algorithms, and provide insight on using such
%% algorithms in practice. We propose a heuristic-based algorithm, Solstice
%% %% , that takes advantage of skew and sparsity in typical datacenter traffic
%% %% demand matrices to
%% that provides a 2.9$\times$ increase in circuit utilization over traditional
%% scheduling algorithms, while being within 14\% of optimal, at scale.

\section{Introduction}
\label{sec:intro}

%% Maybe use some of this from SOLSTICE?
%% Today's datacenters aggregate tremendous amounts of compute and storage
%% capacity, driving demand for network switches with ever-increasing port counts
%% and line speeds.  However, supporting these demands with existing packet
%% switching technology is becoming increasingly expensive---in cost, heat, power,
%% and cabling. Packet switches are flexible, capable of making forwarding
%% decisions at the granularity of individual packets.  In common modern scenarios,
%% however, this flexibility is unnecessary: many (often consecutive) packets are
%% sent to the same output port.  Two key factors contribute to this traffic
%% pattern. First, traffic inside a datacenter often has high spatial locality,
%% where a large fraction of the traffic that enters each switch port is destined
%% for only a small number of output ports~\cite{msft-imc09, facebook:sigcomm15}.
%% Second, traffic is often bursty, with significant temporal locality between
%% packets sharing the same destination~\cite{bullet:conext13,
%%   facebook:sigcomm15}. The consequence of these two factors is that the traffic
%% demand matrix at a datacenter switch is often both skewed and
%% sparse~\cite{mordia:hotnets12,flyways,augmenting-dc-wireless}.

%% \begin{figure*}[t!!!]
%% \centering
%% \includegraphics[width=1\textwidth]{figures/setting}
%% \caption{Our model of a hybrid switch architecture and the scheduling
%%   process. The circuit switch has high bandwidth, but slow reconfiguration
%%   time. The packet switch has low bandwidth (e.g., an order of magnitude lower),
%%   but can make forwarding decisions per-packet.
%%   %Both switches must have high utilization.
%%   }
%% \label{fig:arch}
%% \end{figure*}

%% Researchers have seized upon these observations to propose hybrid datacenter
%% network architectures that offer higher throughput at lower cost by combining
%% high-speed optical~\cite{OSA,helios:sigcomm10,c-Through} or
%% wireless~\cite{flyways,augmenting-dc-wireless,mirror-mirror} circuit switching
%% technologies with traditional electronic packet switches. Typically, the circuit
%% switch has a significantly higher data rate than the packet switch, but incurs a
%% non-trivial reconfiguration penalty. While the potential cost savings that
%% hybrid techniques could realize is large, the design space of scheduling
%% algorithms that enable high utilization in hybrid networks is not yet well
%% understood. Earlier work that considers circuit switches with substantial
%% reconfiguration delay offers no guidance about how to negotiate the trade-off
%% between remaining in the current (potentially sub-optimal) circuit configuration
%% vs. incurring a costly reconfiguration delay to switch to a potentially better
%% circuit configuration~\cite{c-Through, helios:sigcomm10, wang:hotnets}.  The
%% reconfiguration cost of these systems was so high that they were forced to keep
%% a configuration pinned up for a relatively long period anyway.

%% In recent years, however, the switching time of optical circuit switches has
%% improved substantially~\cite{mordia:sigcomm13}. As a result, an efficient
%% scheduling algorithm for a modern hybrid design must determine: 1) a set of
%% circuit configurations (which ports are connected to which other ports and how
%% long that configuration should remain in effect) designed to maximize the
%% traffic serviced over the high-bandwidth but slow-to-reconfigure
%% circuit-switched network, and 2) what traffic should be sent to the
%% low-bandwidth but flexible packet switch.

%% Computing an optimal set of circuit configurations to maximize circuit-switched
%% utilization has no known polynomial-time algorithms, scaling as $O(n!)$ in the
%% number of switch ports (\S\ref{sec:optimization}). The challenge arises due to
%% the non-trivial switching time between configurations, which necessitates not
%% only sending as much traffic as possible, but doing so in the fewest number of
%% configurations.

%% The end goal of this paper is an effective and fast heuristic algorithm that
%% delivers high switch utilization. To this end, we first provide a detailed
%% characterization of the problem plus an optimal (but impractical) solution that
%% sheds light on how to design an effective heuristic.  We then present our
%% heuristic, Solstice, which provides 2.9$\times$ higher utilization compared to
%% previous algorithms by taking advantage of the known sparsity and skew of
%% datacenter workloads---some of the same features that make the traditional
%% scheduling problem hard.

%% The contributions of this work are as follows:
%% \begin{enumerate}
%% \item Characterizing the hybrid switch scheduling problem.
%% \item Exploring the design space of hybrid scheduling:
%%   \begin{packedenumerate}
%%     \item \myparatight{Lower bound} an instantly computable but loose bound on
%%       the minimum amount of time it takes to serve all demand (but provides no
%%       actual schedules).
%%     \item \myparatight{Optimal scheduling} optimally schedule all
%%       demand with minimal time; impossible to run in real time at scale.
%%     \item \myparatight{Heuristic algorithm (``Solstice'')} runs in real time at
%%       scale, but slightly underperforms optimal (by at most 14\%
%%       at target scale).
%%     \item \myparatight{Heuristic + optimization (``\mbox{Solstice{}++}'')} runs at
%%       scale (though not in real time), but tightens the gap between Solstice and
%%       optimal (at most 12\% from optimal at target scale).
%%   \end{packedenumerate}
%% \item Insight into the challenges and benefits of using hybrid switches, with a
%%   focus on high circuit utilization.
%% \end{enumerate}

\tightsection{The Setting}
\label{sec:setting}

\section{Problems of CDN Commoditization}
\label{sec:problems}

\section{Designing a Better Interface}
\label{sec:design}

\tightsection{Related work}
\label{sec:related}


\tightsection{Conclusion}
\label{sec:concl}



\bibliographystyle{acm}
\bibliography{ref}

\end{document}
